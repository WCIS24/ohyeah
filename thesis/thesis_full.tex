\documentclass[UTF8,openany]{ctexrep}

\usepackage{fontspec}
\usepackage{xeCJK}
\usepackage{setspace}
\usepackage{geometry}
\usepackage{booktabs}
\usepackage{longtable}
\usepackage{caption}
\usepackage{hyperref}
\usepackage{amsmath}
\usepackage{tikz}
\usepackage{enumitem}

% 页面设置
\geometry{a4paper, left=2.5cm, right=2.5cm, top=2.5cm, bottom=2.5cm}

% 字体设置(可按学校环境替换)
\setmainfont{Times New Roman}
\setCJKmainfont{SimSun}       % 宋体
\setCJKsansfont{SimHei}       % 黑体
\setCJKfamilyfont{zhongsong}{STZhongsong} % 华文中宋(若无可替换)

% 行距与缩进
\setstretch{1.5}
\setlength{\parindent}{2em}
\raggedright

% 脚注字号
\renewcommand\footnotesize{\zihao{-5}}

% 章节标题格式
\ctexset{
  chapter = {
    format=\centering\heiti\zihao{-2},
    name = {第,章},
    number = \chinese{chapter},
    beforeskip = 1em,
    afterskip = 1em
  },
  section = {
    format=\heiti\zihao{4},
    beforeskip=1em,
    afterskip=0.5em
  },
  subsection = {
    format=\heiti\zihao{-4},
    beforeskip=0.8em,
    afterskip=0.4em
  }
}

% 表题/图题格式
\captionsetup[table]{labelfont={bf},textfont={bf},labelsep=space}
\captionsetup[figure]{labelfont={bf},textfont={bf},labelsep=space}

% 表格行距
\renewcommand{\arraystretch}{1.2}

% 自定义命令
\newcommand{\TODO}{\textbf{【TODO】}}

\begin{document}

\thispagestyle{empty}
\begin{center}
{\CJKfamily{zhongsong}\zihao{-0}\bfseries <论文题目(华文中宋 小初号 加粗)>}\\[2em]
\end{center}

{\songti\zihao{-4}
\noindent 学院:<院系>\\
专业:<专业>\\
班级:<班级>\\
姓名:<姓名>\\
学号:<学号>\\
指导教师:<指导教师>\\
完成日期:<完成日期>\\
}

\cleardoublepage

\thispagestyle{empty}
\begin{center}
{\heiti\zihao{-2}\bfseries 学位论文原创性声明}
\end{center}

{\songti\zihao{-4}
本人郑重声明:所呈交的学位论文是本人在导师指导下独立完成的研究成果。论文中所使用的研究成果及引用内容均已注明出处。本论文不包含任何他人已经发表或撰写过的研究成果。
}

\vspace{3em}
{\songti\zihao{-4}
作者签名:\underline{\hspace{6em}} \hfill 日期:\underline{\hspace{6em}}
}

\cleardoublepage

\thispagestyle{empty}
\begin{center}
{\heiti\zihao{-2}\bfseries 学位论文版权使用授权书}
\end{center}

{\songti\zihao{-4}
本人同意学校在教学与科研范围内保留、使用本论文的权利,包括但不限于复制、汇编、借阅与信息网络传播。
}

\vspace{3em}
{\songti\zihao{-4}
作者签名:\underline{\hspace{6em}} \hfill 日期:\underline{\hspace{6em}}
}

\cleardoublepage

\chapter*{摘要}
{\songti\zihao{-4}
金融问答场景中的复杂查询往往涉及多年份、多实体与显式数值计算,传统单步检索 + 占位式生成在复杂问题上容易出现证据覆盖不足与算术错误。本文面向 FinDER(金融领域专家检索)数据集构建可复现的金融 RAG 系统,在无外部 LLM API 的约束下引入规则驱动的多步检索与显式计算器,并通过门控与系统化调参保证性能稳定。实验表明:检索器微调显著提升整体检索表现(full dev Recall@10: 0.3246→0.3772);多步检索在复杂子集上保持不退化且 MRR 略有提升;计算器门控避免数值指标回退,为后续提升奠定稳定基线。所有实验产出与配置均可在 outputs/ 中追溯与复现。
}

\vspace{1em}
{\heiti\zihao{-4}关键词:}{\songti\zihao{-4}金融问答;检索增强生成;多步检索;显式计算;误差分析;可复现}

\cleardoublepage

\chapter*{Abstract}
{\setmainfont{Times New Roman}\zihao{-4}
Complex financial queries often involve multiple years, entities, and explicit numerical reasoning. Conventional single-step retrieval with template generation tends to suffer from insufficient evidence coverage and arithmetic errors. This thesis builds a reproducible financial RAG system on the FinDER dataset without external LLM APIs, introducing rule-based multi-step retrieval and an explicit calculator with gating. Results show that retriever fine-tuning improves overall retrieval performance (full dev Recall@10: 0.3246→0.3772); multi-step retrieval remains non-degrading on the complex subset with a slight MRR gain; and calculator gating prevents numeric metric regression, providing a stable baseline for future improvement. All artifacts are traceable under outputs/.
}

\vspace{1em}
{\bfseries Key words:} financial QA; retrieval-augmented generation; multi-step retrieval; explicit calculation; error analysis; reproducibility

\cleardoublepage

\renewcommand{\contentsname}{目录}
\tableofcontents

\cleardoublepage

\mainmatter

\cleardoublepage
\chapter{绪论}

\section{研究背景与动机}
金融领域问题往往涉及最新市场数据、法规条文与财报细节,具有高信息密度与强语义压缩特征。传统大模型在金融问答中易出现事实性偏差或知识滞后。为提高准确性,检索增强生成(RAG)通过“检索—生成”机制使回答基于真实证据,但现有金融 RAG 多以单步检索为主,面对复杂查询时表现不足。

复杂查询通常具备以下特征:跨段落/跨文档证据、显式数值计算需求、金融缩写与实体歧义。FinDER 数据集(Financial Domain Expert Retrieval)包含 5,703 个真实金融问答三元组,凸显多步检索与显式推理的必要性。

\section{研究目标}
本文目标是构建一个可复现、可审计的金融 RAG 基线系统,并在此基础上验证多步检索与显式计算模块的价值与边界,形成工程化可复验结果。

\section{主要贡献}
\begin{itemize}
  \item 构建可复现工程框架,统一数据、检索、评测与实验编排,输出可追踪 run\_id。
  \item 设计规则驱动的多步检索循环,包含 gap 检测、合并与停止策略。
  \item 引入显式计算器与门控机制,控制数值错误传播。
  \item 基于 Step6 的系统调参,输出 full dev / complex dev / numeric dev 的对照结果与误差分析。
\end{itemize}

\cleardoublepage
\chapter{相关工作}

\section{金融问答与金融文本理解}
金融 QA 聚焦财报、公告、研报等长文档中的事实与数值问答,强调证据定位与实体对齐。缩写、口径差异与跨段依赖是该领域的主要挑战。

\section{检索增强生成(RAG)评测}
RAG 将检索与生成结合,评测工作关注稀疏/稠密/混合检索组合以及检索器微调对下游任务的传导效果。

\section{多跳/多步检索}
多步检索通过迭代检索补全证据链,常用于复杂比较与跨文档推理任务,核心难点在于“何时继续检索”与“如何合并证据”。

\section{智能化 RAG(Agentic RAG)}
Agentic RAG 通过规划与工具调用提升推理能力,但在成本、可控性与可复现性方面仍存在现实约束。本文采用规则化多步检索以保证稳定性。

\cleardoublepage
\chapter{方法}

\section{系统总体架构}
本文系统采用配置驱动的流水线式设计,整体流程为:数据规范化与分块 → 检索器索引构建 → (可选)多步检索循环 → (可选)证据计算器 → 模板化答案生成。系统核心模块位于 \texttt{src/},脚本入口位于 \texttt{scripts/},所有实验输出统一落盘到 \texttt{outputs/<run\_id>/}。图\ref{fig:arch}给出整体数据流示意。

\begin{figure}[htbp]
\centering
\begin{tikzpicture}[node distance=1.4cm, every node/.style={draw, rounded corners, align=center, minimum height=0.8cm}]
  \node (data) {FinDER\\数据与规范化};
  \node (chunk) [below=of data] {证据分块\\chunk\_size/overlap};
  \node (index) [below=of chunk] {索引构建\\BM25+Dense};
  \node (retrieve) [below=of index] {检索(单步)};
  \node (multistep) [below=of retrieve] {多步检索\\Gap\&Refine};
  \node (calc) [below=of multistep] {显式计算器\\YoY/差值};
  \node (gen) [below=of calc] {模板化答案};

  \draw[->] (data) -- (chunk);
  \draw[->] (chunk) -- (index);
  \draw[->] (index) -- (retrieve);
  \draw[->] (retrieve) -- (multistep);
  \draw[->] (multistep) -- (calc);
  \draw[->] (calc) -- (gen);
\end{tikzpicture}
\caption{基于仓库实现的金融 RAG 系统流程示意图}
\label{fig:arch}
\end{figure}

\section{数据规范化与语料构建}
数据加载与统一格式化由 \texttt{src/data/finder.py} 实现,脚本入口为 \texttt{scripts/prepare\_data.py}。统一样本格式见 \texttt{docs/DATA\_FORMAT.md},核心字段包括 \texttt{qid/query/answer/evidences}。默认切分比例来自 \texttt{configs/prepare\_data.yaml}:train/dev/test = 0.8/0.1/0.1,随机种子为 42。

语料分块由 \texttt{src/indexing/chunking.py} 实现,使用字符级切分并保留 \texttt{chunk\_id} 与来源信息。默认分块参数见 \texttt{configs/build\_corpus.yaml}(\texttt{chunk\_size=1000, overlap=100}),Step6 的系统调参使用 \texttt{configs/step6\_base.yaml}(\texttt{chunk\_size=200, overlap=50})。分块产物写入 \texttt{data/corpus/chunks.jsonl}。

\section{检索器与混合检索}
检索器实现位于 \texttt{src/retrieval/retriever.py},核心类为 \texttt{HybridRetriever}。系统同时构建 BM25 稀疏索引(\texttt{rank\_bm25})与稠密向量索引(\texttt{sentence\_transformers}),并支持 FAISS(若不可用则回退 NumPy brute-force)。

混合检索按查询级进行 min-max 归一化并线性融合(详见 \texttt{docs/METRICS.md}):
\begin{equation}
\tilde{s} = \frac{s-\min(s)}{\max(s)-\min(s)},\quad
s_{hybrid}=\alpha\tilde{s}_{bm25}+(1-\alpha)\tilde{s}_{dense}
\end{equation}

默认检索参数来自 \texttt{configs/run\_baseline.yaml}(\texttt{mode=hybrid, alpha=0.5, top\_k=5}),Step6 统一配置使用 \texttt{configs/step6\_base.yaml}(\texttt{mode=dense},保留 \texttt{alpha=0.5} 以支持混合检索扩展)。

\section{规则驱动多步检索}
多步检索引擎位于 \texttt{src/multistep/engine.py},由以下组件组成:
\begin{itemize}
  \item \textbf{StepPlanner}(\texttt{planner.py}):基于年份、百分号与比较关键词判定查询类型(COMPARE/TREND/FACT/OTHER)。
  \item \textbf{GapDetector}(\texttt{gap.py}):检测年份缺口或比较对象缺口,并输出 \texttt{gap\_conf}。
  \item \textbf{QueryRefiner}(\texttt{refiner.py}):在缺口存在时追加缺失年份/实体或指标同义词。
  \item \textbf{StopCriteria}(\texttt{stop.py}):满足 \texttt{MAX\_STEPS}、\texttt{NO\_GAP}、\texttt{EMPTY\_RESULTS} 或 \texttt{NO\_NEW\_EVIDENCE} 时停止。
  \item \textbf{Merge Strategy}:\texttt{maxscore} 或 \texttt{step1\_first},并最终截断为 \texttt{top\_k\_final}。
\end{itemize}

配置中区分每步检索规模 \texttt{top\_k\_each\_step} 与最终输出规模 \texttt{top\_k\_final}(\texttt{configs/run\_multistep.yaml} 默认 \texttt{top\_k\_each\_step=5, top\_k\_final=10}),避免评测时 Recall@k 被候选数量截断。门控规则由 \texttt{multistep.gate} 控制,Step6 默认阈值为 \texttt{min\_gap\_conf=0.3}(\texttt{configs/step6\_base.yaml})。

\section{证据抽取与显式计算}
数值抽取位于 \texttt{src/calculator/extract.py},通过正则识别数值、年份、百分号与单位,输出 Fact 数据结构(包含 \texttt{value/unit/year/confidence} 等字段)。计算模块位于 \texttt{src/calculator/compute.py},支持 YoY、差值、占比与倍数四类任务;当单位不一致、年份缺失或候选冲突时,结果标记为 \texttt{unit\_mismatch}/\texttt{insufficient\_facts}/\texttt{ambiguous} 并拒算。

计算门控由 \texttt{calculator.gate} 控制(\texttt{configs/step6\_base.yaml}:\texttt{min\_conf=0.4},\texttt{require\_unit\_consistency=true},\texttt{require\_year\_match=true}),并可在 Step6 阈值搜索中调整。

\section{答案生成与可审计输出}
答案生成采用模板化方式:若计算器结果 \texttt{status=ok} 且通过门控,则输出结构化答案与解释;否则回退基线答案并记录 \texttt{fallback\_reason}。主要输出文件包括:
\begin{itemize}
  \item \texttt{retrieval\_results.jsonl}、\texttt{multistep\_traces.jsonl}(多步检索证据与轨迹);
  \item \texttt{facts.jsonl}、\texttt{results\_R.jsonl}、\texttt{calc\_traces.jsonl}(显式计算链路);
  \item \texttt{predictions.jsonl} / \texttt{predictions\_calc.jsonl}(答案预测);
  \item \texttt{metrics.json} / \texttt{numeric\_metrics.json}(评测指标);
  \item \texttt{config.yaml}、\texttt{git\_commit.txt}、\texttt{logs.txt}(可复现与审计)。
\end{itemize}

所有脚本均以 YAML 配置驱动,并在输出目录中写入解析后的 \texttt{config.resolved.yaml} 以记录最终参数。

\cleardoublepage
\chapter{实验设置与结果分析}

\section{数据集与划分}
实验使用 FinDER 数据集,包含 5,703 个查询—证据—答案三元组(见 \texttt{docs/EXPERIMENT.md})。数据由 \texttt{scripts/prepare\_data.py} 规范化后划分为 train/dev/test(\texttt{configs/prepare\_data.yaml}:0.8/0.1/0.1),统一字段格式见 \texttt{docs/DATA\_FORMAT.md}。

\section{子集构建}
复杂子集 \texttt{complex\_dev} 的构建规则来自 \texttt{scripts/build\_subsets.py} 与 \texttt{configs/build\_subsets.yaml}:满足“多证据”“含≥2年份”“含比较关键词”“含年份且含数字/百分号”等条件之一即进入子集。数值子集 \texttt{numeric\_dev} 来自 \texttt{scripts/build\_numeric\_subset.py},规则包括查询/答案含数字、含百分号或同比/差值/倍数关键词等。子集列表写入 \texttt{data/subsets/}。

\section{评价指标}
检索与 QA 指标定义见 \texttt{docs/METRICS.md},本文使用:
\begin{itemize}
  \item 检索指标:Recall@k、MRR@k、evidence\_hit@k;
  \item QA 指标:EM/F1(对照);
  \item 数值指标:Numeric-EM、RelErr、Coverage(\texttt{scripts/eval\_numeric.py})。
\end{itemize}
当缺失 \texttt{doc\_id/evidence\_id} 时采用文本包含匹配,并记录 \texttt{uncertain\_match\_ratio}。

\section{实验设置与实现细节}
实验依赖写在 \texttt{requirements.txt},包括 \texttt{datasets}、\texttt{rank-bm25}、\texttt{sentence-transformers} 等。默认随机种子为 42(见各配置文件)。FAISS 若不可用则回退 NumPy brute-force(见 README 与 \texttt{src/retrieval/retriever.py})。

Step6 统一配置位于 \texttt{configs/step6\_base.yaml},关键参数包括:\texttt{chunk\_size=200}、\texttt{overlap=50}、\texttt{multistep.max\_steps=3}、\texttt{top\_k\_each\_step=5}、\texttt{multistep.merge\_strategy=maxscore}、\texttt{calculator.gate.min\_conf=0.4} 等。Step6 组合实验矩阵与最优配置见 \texttt{configs/step6\_matrix.yaml} 与 \texttt{configs/step6\_experiments.yaml}。

\section{检索器微调结果(Step3)}
表\ref{tab:ft} 给出 Step3 检索器微调前后在 dev 集的检索指标(\texttt{docs/RESULTS\_RETRIEVER\_FT.md})。
\begin{table}[htbp]
\centering
\caption{检索器微调前后对比(dev)}
\label{tab:ft}
\begin{tabular}{lcccccc}
\toprule
设置 & Recall@1 & Recall@5 & Recall@10 & MRR@1 & MRR@5 & MRR@10 \\
\midrule
pre\_ft & 0.156140 & 0.259649 & 0.324561 & 0.156140 & 0.194327 & 0.202958 \\
post\_ft & 0.208772 & 0.331579 & 0.377193 & 0.208772 & 0.253977 & 0.260061 \\
\bottomrule
\end{tabular}
\end{table}

\section{多步检索结果与问题定位(Step4)}
Step4 早期结果显示多步检索在 full dev 上的 Recall@10 低于 baseline,并与 Recall@5 相同(见 \texttt{docs/RESULTS\_MULTISTEP.md})。该现象来自候选数量被 \texttt{top\_k\_each\_step} 截断,后续已通过 \texttt{top\_k\_final} 修复(\texttt{configs/run\_multistep.yaml})。修复后所有评测均保证 \texttt{top\_k\_final} 不小于 \texttt{k\_values} 的最大值。

\section{系统化调参与最终对照(Step6)}
表\ref{tab:main} 汇总 Step6 的 6 组组合实验(full dev 与 complex dev 指标),数据来源 \texttt{docs/TABLE\_MAIN.md}。表\ref{tab:numeric} 汇总 numeric dev 的数值指标(\texttt{docs/TABLE\_NUMERIC.md})。

\begin{table}[htbp]
\centering
\caption{Step6 主结果(full dev / complex dev)}
\label{tab:main}
\begin{tabular}{lccccc}
\toprule
label & full\_r10 & full\_mrr10 & complex\_r10 & complex\_mrr10 \\
\midrule
pre\_ft\_baseline & 0.3246 & 0.2030 & 0.3457 & 0.2330 \\
post\_ft\_baseline & 0.3772 & 0.2601 & 0.3909 & 0.2960 \\
post\_ft\_multistep\_best & 0.3772 & 0.2601 & 0.3909 & 0.2961 \\
post\_ft\_baseline\_calc\_best & 0.3772 & 0.2601 & 0.3909 & 0.2960 \\
post\_ft\_multistep\_calc\_best & 0.3772 & 0.2601 & 0.3909 & 0.2961 \\
post\_ft\_multistep\_T1\_calc\_best & 0.3772 & 0.2601 & 0.3909 & 0.2960 \\
\bottomrule
\end{tabular}
\end{table}

\begin{table}[htbp]
\centering
\caption{Step6 数值子集结果(numeric dev)}
\label{tab:numeric}
\begin{tabular}{lccc}
\toprule
label & num\_em & num\_rel & num\_cov \\
\midrule
pre\_ft\_baseline & 0.3791 & 2874.5248 & 0.6202 \\
post\_ft\_baseline & 0.3838 & 683.3536 & 0.6266 \\
post\_ft\_multistep\_best & -- & -- & -- \\
post\_ft\_baseline\_calc\_best & 0.3838 & 683.3536 & 0.6266 \\
post\_ft\_multistep\_calc\_best & 0.3838 & 683.3536 & 0.6266 \\
post\_ft\_multistep\_T1\_calc\_best & 0.3838 & 683.3536 & 0.6266 \\
\bottomrule
\end{tabular}
\end{table}

\section{计算器影响(Step5)}
Step5 记录了未门控情况下的计算器对数值指标的影响(\texttt{docs/RESULTS\_CALCULATOR.md})。numeric\_dev 上 baseline 的 Numeric-EM 为 0.3617,baseline+calculator 为 0.3106,multistep+calculator 为 0.3038,表明在缺乏强门控时覆盖率提升但误差上升。该观察推动了 Step6 的门控与阈值搜索策略。

\section{误差分析与案例}
错误类型汇总见 \texttt{docs/ERROR\_ANALYSIS.md}。complex\_dev 的主要失败集中在 \texttt{no\_gap} 与 \texttt{max\_steps},说明部分查询在首步已覆盖核心证据,而 refined query 的补充作用有限。numeric\_dev 中由于计算器门控较保守,fallback 占比高。

典型复杂查询案例详见 \texttt{docs/CASE\_STUDIES.md}。例如 qid=8c8c8c34 的多年份对比问题在多步检索下识别到 gap 但新增证据有限;qid=ed746c33 则在首步即覆盖核心证据,多步检索未引入噪声。

\section{可复现性与运行步骤}
所有实验通过脚本与 Makefile 驱动,关键命令如下(与 README 一致):
\begin{verbatim}
# 数据准备
python scripts\prepare_data.py --config configs\prepare_data.yaml
python scripts\build_corpus.py --config configs\build_corpus.yaml

# 检索评测与基线
python scripts\eval_retrieval.py --config configs\eval_retrieval.yaml
python scripts\run_baseline.py --config configs\run_baseline.yaml
python scripts\eval_qa.py --config configs\eval_qa.yaml --predictions outputs/<run_id>/predictions.jsonl --gold data/processed/dev.jsonl

# 多步检索
python scripts\run_multistep_retrieval.py --config configs\run_multistep.yaml
python scripts\eval_multistep_retrieval.py --config configs\eval_multistep.yaml --results outputs/<run_id>/retrieval_results.jsonl

# 计算器与数值评测
python scripts\build_numeric_subset.py --config configs\build_numeric_subset.yaml
python scripts\run_with_calculator.py --config configs\run_with_calculator.yaml
python scripts\eval_numeric.py --config configs\eval_numeric.yaml --predictions outputs/<run_id>/predictions_calc.jsonl

# Step6 调参与表格
python scripts\sweep.py --base-config configs\step6_base.yaml --search-space configs\search_space_multistep.yaml
python scripts\sweep.py --base-config configs\step6_base.yaml --search-space configs\search_space_calc.yaml
python scripts\run_matrix_step6.py --base-config configs\step6_base.yaml --matrix configs\step6_matrix.yaml
python scripts\make_tables.py --experiments configs\step6_experiments.yaml
\end{verbatim}

每次运行都会在 \texttt{outputs/<run\_id>/} 写入配置快照、日志、指标与中间产物,支持完整追溯与复现。

\cleardoublepage
\chapter{结论与展望}

\section{结论}
本文基于仓库 WCIS24/ohyeah 的可复现工程实现了金融 RAG 的完整流水线,并在此基础上验证了多步检索与显式计算的工程价值。实验结果表明:
\begin{itemize}
  \item 检索器微调显著提升检索表现(Recall@10:0.324561 提升至 0.377193)。
  \item 系统化调参后,多步检索在 complex\_dev 上保持不退化,MRR 略有提升(见表\ref{tab:main})。
  \item 计算器在未门控时可能带来数值指标回退,Step6 通过门控与阈值策略保持稳定性(见表\ref{tab:numeric} 与 \texttt{docs/RESULTS\_CALCULATOR.md})。
\end{itemize}

整体而言,本文实现了一个“可复现、可审计、可扩展”的金融 RAG 基线框架,并为后续进一步提升复杂查询与数值推理能力打下工程基础。

\section{展望}
未来工作可从以下方向展开:
\begin{itemize}
  \item \textbf{查询理解增强}:引入更强的金融实体识别与缩写消歧模块,提升 gap 检测的准确性。
  \item \textbf{多步检索策略优化}:在不引入外部 API 的前提下,改进 QueryRefiner 的重写策略与停止准则,减少检索漂移。
  \item \textbf{计算器扩展}:完善单位与口径对齐规则,扩展到更多金融指标的显式计算。
  \item \textbf{评测与案例扩展}:构建更细粒度的错误类型与案例库,提升可解释性分析深度。
\end{itemize}

\cleardoublepage
\chapter*{参考文献}
\addcontentsline{toc}{chapter}{参考文献}

\begin{thebibliography}{99}
\bibitem{todo} \TODO{仓库未提供参考文献条目与 BibTeX 文件,请补充与本文相关的金融 QA、RAG、多步检索、显式计算等文献。}
\end{thebibliography}

\cleardoublepage
\chapter*{附录}
\addcontentsline{toc}{chapter}{附录}

\section*{附录A 复现清单}
\begin{itemize}
  \item \textbf{环境依赖}:见 \texttt{requirements.txt}(numpy、pandas、datasets、rank-bm25、sentence-transformers、pytest 等)。\TODO{需补充 Python 版本、CUDA/GPU 规格(若使用)。}
  \item \textbf{数据获取}:根据 \texttt{configs/prepare\_data.yaml} 需提供 \texttt{dataset/finder\_dataset.csv};smoke 测试可从 \texttt{Linq-AI-Research/FinDER} 自动下载小样本(见 README)。
  \item \textbf{一键命令}:Makefile 目标包括 \texttt{make setup}、\texttt{make smoke}、\texttt{make run\_multistep}、\texttt{make eval\_multistep}、\texttt{make run\_baseline\_calc}、\texttt{make eval\_numeric}、\texttt{make run\_matrix\_step6}、\texttt{make make\_tables}。
  \item \textbf{输出产物}:所有实验产物写入 \texttt{outputs/<run\_id>/},包含配置快照、日志、指标、追踪文件与中间结果。
\end{itemize}

\section*{附录B 需要补充的信息清单}
\begin{itemize}
  \item \TODO{封面字段:学院/专业/班级/姓名/学号/指导教师/完成日期。}
  \item \TODO{参考文献条目与引用格式(若有学校模板或 BibTeX 文件)。}
  \item \TODO{运行环境说明:Python 版本、操作系统、硬件信息。}
  \item \TODO{若需提交学校模板的样式文件,请提供对应 LaTeX/Word 模板要求。}
\end{itemize}

\cleardoublepage
\chapter*{致谢}
\addcontentsline{toc}{chapter}{致谢}

\TODO{请补充致谢内容,例如对导师、同学和家人的感谢。}

\end{document}
