\cleardoublepage
\chapter{方法}

\section{系统总体架构}
本文系统采用配置驱动的流水线式设计,整体流程为:数据规范化与分块 → 检索器索引构建 → (可选)多步检索循环 → (可选)证据计算器 → 模板化答案生成。系统核心模块位于 \texttt{src/},脚本入口位于 \texttt{scripts/},所有实验输出统一落盘到 \texttt{outputs/<run\_id>/}。图\ref{fig:arch}给出整体数据流示意。

\begin{figure}[htbp]
\centering
\begin{tikzpicture}[node distance=1.4cm, every node/.style={draw, rounded corners, align=center, minimum height=0.8cm}]
  \node (data) {FinDER\\数据与规范化};
  \node (chunk) [below=of data] {证据分块\\chunk\_size/overlap};
  \node (index) [below=of chunk] {索引构建\\BM25+Dense};
  \node (retrieve) [below=of index] {检索(单步)};
  \node (multistep) [below=of retrieve] {多步检索\\Gap\&Refine};
  \node (calc) [below=of multistep] {显式计算器\\YoY/差值};
  \node (gen) [below=of calc] {模板化答案};

  \draw[->] (data) -- (chunk);
  \draw[->] (chunk) -- (index);
  \draw[->] (index) -- (retrieve);
  \draw[->] (retrieve) -- (multistep);
  \draw[->] (multistep) -- (calc);
  \draw[->] (calc) -- (gen);
\end{tikzpicture}
\caption{基于仓库实现的金融 RAG 系统流程示意图}
\label{fig:arch}
\end{figure}

\section{数据规范化与语料构建}
数据加载与统一格式化由 \texttt{src/data/finder.py} 实现,脚本入口为 \texttt{scripts/prepare\_data.py}。统一样本格式见 \texttt{docs/DATA\_FORMAT.md},核心字段包括 \texttt{qid/query/answer/evidences}。默认切分比例来自 \texttt{configs/prepare\_data.yaml}:train/dev/test = 0.8/0.1/0.1,随机种子为 42。

语料分块由 \texttt{src/indexing/chunking.py} 实现,使用字符级切分并保留 \texttt{chunk\_id} 与来源信息。默认分块参数见 \texttt{configs/build\_corpus.yaml}(\texttt{chunk\_size=1000, overlap=100}),Step6 的系统调参使用 \texttt{configs/step6\_base.yaml}(\texttt{chunk\_size=200, overlap=50})。分块产物写入 \texttt{data/corpus/chunks.jsonl}。

\section{检索器与混合检索}
检索器实现位于 \texttt{src/retrieval/retriever.py},核心类为 \texttt{HybridRetriever}。系统同时构建 BM25 稀疏索引(\texttt{rank\_bm25})与稠密向量索引(\texttt{sentence\_transformers}),并支持 FAISS(若不可用则回退 NumPy brute-force)。

混合检索按查询级进行 min-max 归一化并线性融合(详见 \texttt{docs/METRICS.md}):
\begin{equation}
\tilde{s} = \frac{s-\min(s)}{\max(s)-\min(s)},\quad
s_{hybrid}=\alpha\tilde{s}_{bm25}+(1-\alpha)\tilde{s}_{dense}
\end{equation}

默认检索参数来自 \texttt{configs/run\_baseline.yaml}(\texttt{mode=hybrid, alpha=0.5, top\_k=5}),Step6 统一配置使用 \texttt{configs/step6\_base.yaml}(\texttt{mode=dense},保留 \texttt{alpha=0.5} 以支持混合检索扩展)。

\section{规则驱动多步检索}
多步检索引擎位于 \texttt{src/multistep/engine.py},由以下组件组成:
\begin{itemize}
  \item \textbf{StepPlanner}(\texttt{planner.py}):基于年份、百分号与比较关键词判定查询类型(COMPARE/TREND/FACT/OTHER)。
  \item \textbf{GapDetector}(\texttt{gap.py}):检测年份缺口或比较对象缺口,并输出 \texttt{gap\_conf}。
  \item \textbf{QueryRefiner}(\texttt{refiner.py}):在缺口存在时追加缺失年份/实体或指标同义词。
  \item \textbf{StopCriteria}(\texttt{stop.py}):满足 \texttt{MAX\_STEPS}、\texttt{NO\_GAP}、\texttt{EMPTY\_RESULTS} 或 \texttt{NO\_NEW\_EVIDENCE} 时停止。
  \item \textbf{Merge Strategy}:\texttt{maxscore} 或 \texttt{step1\_first},并最终截断为 \texttt{top\_k\_final}。
\end{itemize}

配置中区分每步检索规模 \texttt{top\_k\_each\_step} 与最终输出规模 \texttt{top\_k\_final}(\texttt{configs/run\_multistep.yaml} 默认 \texttt{top\_k\_each\_step=5, top\_k\_final=10}),避免评测时 Recall@k 被候选数量截断。门控规则由 \texttt{multistep.gate} 控制,Step6 默认阈值为 \texttt{min\_gap\_conf=0.3}(\texttt{configs/step6\_base.yaml})。

\section{证据抽取与显式计算}
数值抽取位于 \texttt{src/calculator/extract.py},通过正则识别数值、年份、百分号与单位,输出 Fact 数据结构(包含 \texttt{value/unit/year/confidence} 等字段)。计算模块位于 \texttt{src/calculator/compute.py},支持 YoY、差值、占比与倍数四类任务;当单位不一致、年份缺失或候选冲突时,结果标记为 \texttt{unit\_mismatch}/\texttt{insufficient\_facts}/\texttt{ambiguous} 并拒算。

计算门控由 \texttt{calculator.gate} 控制(\texttt{configs/step6\_base.yaml}:\texttt{min\_conf=0.4},\texttt{require\_unit\_consistency=true},\texttt{require\_year\_match=true}),并可在 Step6 阈值搜索中调整。

\section{答案生成与可审计输出}
答案生成采用模板化方式:若计算器结果 \texttt{status=ok} 且通过门控,则输出结构化答案与解释;否则回退基线答案并记录 \texttt{fallback\_reason}。主要输出文件包括:
\begin{itemize}
  \item \texttt{retrieval\_results.jsonl}、\texttt{multistep\_traces.jsonl}(多步检索证据与轨迹);
  \item \texttt{facts.jsonl}、\texttt{results\_R.jsonl}、\texttt{calc\_traces.jsonl}(显式计算链路);
  \item \texttt{predictions.jsonl} / \texttt{predictions\_calc.jsonl}(答案预测);
  \item \texttt{metrics.json} / \texttt{numeric\_metrics.json}(评测指标);
  \item \texttt{config.yaml}、\texttt{git\_commit.txt}、\texttt{logs.txt}(可复现与审计)。
\end{itemize}

所有脚本均以 YAML 配置驱动,并在输出目录中写入解析后的 \texttt{config.resolved.yaml} 以记录最终参数。
