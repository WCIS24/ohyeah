\cleardoublepage
\chapter{绪论}

\section{研究背景与动机}
金融领域问题往往涉及最新市场数据、法规条文与财报细节,具有高信息密度与强语义压缩特征。传统大模型在金融问答中易出现事实性偏差或知识滞后。为提高准确性,检索增强生成(RAG)通过“检索—生成”机制使回答基于真实证据,但现有金融 RAG 多以单步检索为主,面对复杂查询时表现不足。

复杂查询通常具备以下特征:跨段落/跨文档证据、显式数值计算需求、金融缩写与实体歧义。FinDER 数据集(Financial Domain Expert Retrieval)包含 5,703 个真实金融问答三元组,凸显多步检索与显式推理的必要性。

\section{研究目标}
本文目标是构建一个可复现、可审计的金融 RAG 基线系统,并在此基础上验证多步检索与显式计算模块的价值与边界,形成工程化可复验结果。

\section{主要贡献}
\begin{itemize}
  \item 构建可复现工程框架,统一数据、检索、评测与实验编排,输出可追踪 run\_id。
  \item 设计规则驱动的多步检索循环,包含 gap 检测、合并与停止策略。
  \item 引入显式计算器与门控机制,控制数值错误传播。
  \item 基于 Step6 的系统调参,输出 full dev / complex dev / numeric dev 的对照结果与误差分析。
\end{itemize}
