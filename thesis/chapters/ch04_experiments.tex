\cleardoublepage
\chapter{实验设置与结果分析}

\section{数据集与划分}
实验使用 FinDER 数据集,包含 5,703 个查询—证据—答案三元组(见 \texttt{docs/EXPERIMENT.md})。数据由 \texttt{scripts/prepare\_data.py} 规范化后划分为 train/dev/test(\texttt{configs/prepare\_data.yaml}:0.8/0.1/0.1),统一字段格式见 \texttt{docs/DATA\_FORMAT.md}。

\section{子集构建}
复杂子集 \texttt{complex\_dev} 的构建规则来自 \texttt{scripts/build\_subsets.py} 与 \texttt{configs/build\_subsets.yaml}:满足“多证据”“含≥2年份”“含比较关键词”“含年份且含数字/百分号”等条件之一即进入子集。数值子集 \texttt{numeric\_dev} 来自 \texttt{scripts/build\_numeric\_subset.py},规则包括查询/答案含数字、含百分号或同比/差值/倍数关键词等。子集列表写入 \texttt{data/subsets/}。

\section{评价指标}
检索与 QA 指标定义见 \texttt{docs/METRICS.md},本文使用:
\begin{itemize}
  \item 检索指标:Recall@k、MRR@k、evidence\_hit@k;
  \item QA 指标:EM/F1(对照);
  \item 数值指标:Numeric-EM、RelErr、Coverage(\texttt{scripts/eval\_numeric.py})。
\end{itemize}
当缺失 \texttt{doc\_id/evidence\_id} 时采用文本包含匹配,并记录 \texttt{uncertain\_match\_ratio}。

\section{实验设置与实现细节}
实验依赖写在 \texttt{requirements.txt},包括 \texttt{datasets}、\texttt{rank-bm25}、\texttt{sentence-transformers} 等。默认随机种子为 42(见各配置文件)。FAISS 若不可用则回退 NumPy brute-force(见 README 与 \texttt{src/retrieval/retriever.py})。

Step6 统一配置位于 \texttt{configs/step6\_base.yaml},关键参数包括:\texttt{chunk\_size=200}、\texttt{overlap=50}、\texttt{multistep.max\_steps=3}、\texttt{top\_k\_each\_step=5}、\texttt{multistep.merge\_strategy=maxscore}、\texttt{calculator.gate.min\_conf=0.4} 等。Step6 组合实验矩阵与最优配置见 \texttt{configs/step6\_matrix.yaml} 与 \texttt{configs/step6\_experiments.yaml}。

\section{检索器微调结果(Step3)}
表\ref{tab:ft} 给出 Step3 检索器微调前后在 dev 集的检索指标(\texttt{docs/RESULTS\_RETRIEVER\_FT.md})。
\begin{table}[htbp]
\centering
\caption{检索器微调前后对比(dev)}
\label{tab:ft}
\begin{tabular}{lcccccc}
\toprule
设置 & Recall@1 & Recall@5 & Recall@10 & MRR@1 & MRR@5 & MRR@10 \\
\midrule
pre\_ft & 0.156140 & 0.259649 & 0.324561 & 0.156140 & 0.194327 & 0.202958 \\
post\_ft & 0.208772 & 0.331579 & 0.377193 & 0.208772 & 0.253977 & 0.260061 \\
\bottomrule
\end{tabular}
\end{table}

\section{多步检索结果与问题定位(Step4)}
Step4 早期结果显示多步检索在 full dev 上的 Recall@10 低于 baseline,并与 Recall@5 相同(见 \texttt{docs/RESULTS\_MULTISTEP.md})。该现象来自候选数量被 \texttt{top\_k\_each\_step} 截断,后续已通过 \texttt{top\_k\_final} 修复(\texttt{configs/run\_multistep.yaml})。修复后所有评测均保证 \texttt{top\_k\_final} 不小于 \texttt{k\_values} 的最大值。

\section{系统化调参与最终对照(Step6)}
表\ref{tab:main} 汇总 Step6 的 6 组组合实验(full dev 与 complex dev 指标),数据来源 \texttt{docs/TABLE\_MAIN.md}。表\ref{tab:numeric} 汇总 numeric dev 的数值指标(\texttt{docs/TABLE\_NUMERIC.md})。

\begin{table}[htbp]
\centering
\caption{Step6 主结果(full dev / complex dev)}
\label{tab:main}
\begin{tabular}{lccccc}
\toprule
label & full\_r10 & full\_mrr10 & complex\_r10 & complex\_mrr10 \\
\midrule
pre\_ft\_baseline & 0.3246 & 0.2030 & 0.3457 & 0.2330 \\
post\_ft\_baseline & 0.3772 & 0.2601 & 0.3909 & 0.2960 \\
post\_ft\_multistep\_best & 0.3772 & 0.2601 & 0.3909 & 0.2961 \\
post\_ft\_baseline\_calc\_best & 0.3772 & 0.2601 & 0.3909 & 0.2960 \\
post\_ft\_multistep\_calc\_best & 0.3772 & 0.2601 & 0.3909 & 0.2961 \\
post\_ft\_multistep\_T1\_calc\_best & 0.3772 & 0.2601 & 0.3909 & 0.2960 \\
\bottomrule
\end{tabular}
\end{table}

\begin{table}[htbp]
\centering
\caption{Step6 数值子集结果(numeric dev)}
\label{tab:numeric}
\begin{tabular}{lccc}
\toprule
label & num\_em & num\_rel & num\_cov \\
\midrule
pre\_ft\_baseline & 0.3791 & 2874.5248 & 0.6202 \\
post\_ft\_baseline & 0.3838 & 683.3536 & 0.6266 \\
post\_ft\_multistep\_best & -- & -- & -- \\
post\_ft\_baseline\_calc\_best & 0.3838 & 683.3536 & 0.6266 \\
post\_ft\_multistep\_calc\_best & 0.3838 & 683.3536 & 0.6266 \\
post\_ft\_multistep\_T1\_calc\_best & 0.3838 & 683.3536 & 0.6266 \\
\bottomrule
\end{tabular}
\end{table}

\section{计算器影响(Step5)}
Step5 记录了未门控情况下的计算器对数值指标的影响(\texttt{docs/RESULTS\_CALCULATOR.md})。numeric\_dev 上 baseline 的 Numeric-EM 为 0.3617,baseline+calculator 为 0.3106,multistep+calculator 为 0.3038,表明在缺乏强门控时覆盖率提升但误差上升。该观察推动了 Step6 的门控与阈值搜索策略。

\section{误差分析与案例}
错误类型汇总见 \texttt{docs/ERROR\_ANALYSIS.md}。complex\_dev 的主要失败集中在 \texttt{no\_gap} 与 \texttt{max\_steps},说明部分查询在首步已覆盖核心证据,而 refined query 的补充作用有限。numeric\_dev 中由于计算器门控较保守,fallback 占比高。

典型复杂查询案例详见 \texttt{docs/CASE\_STUDIES.md}。例如 qid=8c8c8c34 的多年份对比问题在多步检索下识别到 gap 但新增证据有限;qid=ed746c33 则在首步即覆盖核心证据,多步检索未引入噪声。

\section{可复现性与运行步骤}
所有实验通过脚本与 Makefile 驱动,关键命令如下(与 README 一致):
\begin{verbatim}
# 数据准备
python scripts\prepare_data.py --config configs\prepare_data.yaml
python scripts\build_corpus.py --config configs\build_corpus.yaml

# 检索评测与基线
python scripts\eval_retrieval.py --config configs\eval_retrieval.yaml
python scripts\run_baseline.py --config configs\run_baseline.yaml
python scripts\eval_qa.py --config configs\eval_qa.yaml --predictions outputs/<run_id>/predictions.jsonl --gold data/processed/dev.jsonl

# 多步检索
python scripts\run_multistep_retrieval.py --config configs\run_multistep.yaml
python scripts\eval_multistep_retrieval.py --config configs\eval_multistep.yaml --results outputs/<run_id>/retrieval_results.jsonl

# 计算器与数值评测
python scripts\build_numeric_subset.py --config configs\build_numeric_subset.yaml
python scripts\run_with_calculator.py --config configs\run_with_calculator.yaml
python scripts\eval_numeric.py --config configs\eval_numeric.yaml --predictions outputs/<run_id>/predictions_calc.jsonl

# Step6 调参与表格
python scripts\sweep.py --base-config configs\step6_base.yaml --search-space configs\search_space_multistep.yaml
python scripts\sweep.py --base-config configs\step6_base.yaml --search-space configs\search_space_calc.yaml
python scripts\run_matrix_step6.py --base-config configs\step6_base.yaml --matrix configs\step6_matrix.yaml
python scripts\make_tables.py --experiments configs\step6_experiments.yaml
\end{verbatim}

每次运行都会在 \texttt{outputs/<run\_id>/} 写入配置快照、日志、指标与中间产物,支持完整追溯与复现。
