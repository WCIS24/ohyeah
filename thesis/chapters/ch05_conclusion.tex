\cleardoublepage
\chapter{结论与展望}

\section{结论}
本文基于仓库 WCIS24/ohyeah 的可复现工程实现了金融 RAG 的完整流水线,并在此基础上验证了多步检索与显式计算的工程价值。实验结果表明:
\begin{itemize}
  \item 检索器微调显著提升检索表现(Recall@10:0.324561 提升至 0.377193)。
  \item 系统化调参后,多步检索在 complex\_dev 上保持不退化,MRR 略有提升(见表\ref{tab:main})。
  \item 计算器在未门控时可能带来数值指标回退,Step6 通过门控与阈值策略保持稳定性(见表\ref{tab:numeric} 与 \texttt{docs/RESULTS\_CALCULATOR.md})。
\end{itemize}

整体而言,本文实现了一个“可复现、可审计、可扩展”的金融 RAG 基线框架,并为后续进一步提升复杂查询与数值推理能力打下工程基础。

\section{展望}
未来工作可从以下方向展开:
\begin{itemize}
  \item \textbf{查询理解增强}:引入更强的金融实体识别与缩写消歧模块,提升 gap 检测的准确性。
  \item \textbf{多步检索策略优化}:在不引入外部 API 的前提下,改进 QueryRefiner 的重写策略与停止准则,减少检索漂移。
  \item \textbf{计算器扩展}:完善单位与口径对齐规则,扩展到更多金融指标的显式计算。
  \item \textbf{评测与案例扩展}:构建更细粒度的错误类型与案例库,提升可解释性分析深度。
\end{itemize}
