\documentclass[UTF8]{ctexart}
\usepackage{geometry}
\usepackage{booktabs}
\usepackage{amsmath}
\usepackage{longtable}
\geometry{a4paper,left=2.5cm,right=2.5cm,top=2.5cm,bottom=2.5cm}
\title{????}
\author{???????}
\date{\today}
\begin{document}
\maketitle
# 金融RAG毕业论文(定稿草稿)

\section{摘要}

金融问答(FinDER)任务中的复杂查询常涉及多年份、多实体与显式数值计算。传统单步检索+占位式生成在复杂问题上易出现证据覆盖不足与算术错误。本文构建了一个可复现的金融RAG系统,在无外部LLM API的约束下,引入规则驱动的多步检索与显式计算器,并通过门控策略与系统化调参避免性能回退。实验结果表明:检索器微调显著提升整体检索表现(full dev Recall@10: 0.3246→0.3772);多步检索在复杂子集上维持不退化且MRR略有提升;计算器通过门控确保数值指标不下降,为后续提升奠定稳定基线。

\textbf{关键词}:金融问答;检索增强生成;多步检索;显式计算;误差分析;可复现

---

\section{1 引言}

金融问答(FinDER)场景中的查询往往具有\textbf{高信息密度}与\textbf{强对比/计算需求}:同一问题可能同时涉及多个年份、实体、指标,并要求对证据进行对齐与推理。传统单步检索 + 占位式生成容易在复杂问题上出现\textbf{证据覆盖不足}与\textbf{算术错误}。

为此,本文围绕可复现的金融 RAG(Retrieval-Augmented Generation)系统,构建并验证了一个分层可控的工程方案:在强约束(无外部 LLM API)的条件下,引入\textbf{规则驱动的多步检索}与\textbf{显式计算器},并通过系统化调参与门控机制避免性能回退。

本文贡献如下:

\begin{itemize}
\item \textbf{可复现工程框架}:建立从数据规范化、检索、评测到实验编排的一体化体系,所有实验产出统一落盘,可审计、可回滚。
\item \textbf{多步检索与门控策略}:实现 gap 检测、合并策略与停止规则,构建可控的多步检索循环,保障复杂查询的证据覆盖。
\item \textbf{显式计算器}:在证据抽取与单位/年份校验基础上进行程序化计算,并通过门控阈值降低算错风险。
\item \textbf{系统化调参与对照分析}:输出 full dev / complex dev / numeric dev 的对照与消融结果,支持论文级结果表格与错误分析。
\end{itemize}

> 说明:请求中提到的 `my-thesis/baseline.pdf` 在当前仓库未找到,本文结构参考现有 Step6 结果与工程文档完成。

---

\section{2 相关工作}

本研究与以下方向密切相关:

\section{金融问答与金融文本理解}
金融 QA 数据集与任务通常聚焦于财报、公告、研报等长文档环境中的事实与数值问答。FinDER 等数据集强调对证据的精确定位与多字段对齐,对检索与推理能力要求较高。

\section{检索增强生成(RAG)}
RAG 框架通过检索模块提供高相关证据,再由生成模块构造答案。近期研究关注\textbf{密集检索、稀疏检索与混合检索}的结合,以及检索器微调对下游 QA 的传导效果。

\section{多跳/多步检索与推理}
多跳检索强调通过多轮检索逐步完善证据覆盖,常见于复杂比较问题与跨文档推理问题。本文的多步检索采用规则驱动的 gap 检测与可控门控策略,以保证解释性和稳定性。

\section{显式数值推理与计算}
数值类问题的核心挑战是单位、年份与数值的对齐。显式计算器通过结构化抽取与程序化计算降低算术错误,并配合门控策略避免错误传播。

---

\section{3 方法}

本文系统由四个核心模块组成:\textbf{查询理解}、\textbf{多步检索推理}、\textbf{证据整合与计算}、\textbf{答案生成}。整体流程如下:

\begin{verbatim}
flowchart TD
  Q[Query] --> U[Query Understanding]
  U --> R1[Step-1 Retrieval]
  R1 --> G[Gap Detector]
  G -->|gap & gate| Rn[Refined Query Retrieval]
  Rn --> M[Merge & Rank]
  M --> C[Calculator (optional)]
  C --> A[Template Answer]
  G -->|no gap| M
\end{verbatim}

\section{1. 查询理解}
通过规则与词典对查询进行初步解析,包括年份识别、比较关系识别、数值提示词识别等。该模块用于驱动多步检索的 gap 类型判断与后续计算任务类型预测。

\section{2. 多步检索推理}
多步检索在每一步使用当前查询进行检索,随后基于 gap 检测决定是否继续检索。核心机制包括:

\begin{itemize}
\item \textbf{Gap Detector}:检测年份缺失、实体缺失等信息缺口。
\item \textbf{Gate(门控)}:当 gap_conf < min_gap_conf 时,停止后续检索(避免 query 漂移)。
\item \textbf{Merge Strategy}:采用 `maxscore` 或 `step1_first` 合并策略对跨步候选排序与截断。
\item \textbf{Stop Criteria}:达到 max_steps 或无新增证据时停止。
\end{itemize}

\textbf{Step6 最优多步配置(dev)}:
\begin{itemize}
\item max_steps=2
\item top_k_each_step=10
\item novelty_threshold=0.0
\item stop_no_new_steps=1
\item merge_strategy=maxscore
\item gate.min_gap_conf=0.3
\end{itemize}

\section{3. 证据整合与计算}
该模块包含数值抽取与显式计算:

\begin{itemize}
\item \textbf{抽取}:从证据文本中提取数值、单位、年份,并标注 inferred_year 与 confidence。
\item \textbf{计算器}:支持 YoY / 差值 / 占比 / 倍数等任务。对于单位不一致、年份缺失、候选冲突等情况进行拒算,并记录原因。
\end{itemize}

\textbf{Step6 最优门控(numeric dev)}:
\begin{itemize}
\item min_conf=0.2
\item allow_task_types=[](在当前版本中关闭计算任务以避免 Numeric-EM 回退)
\end{itemize}

\section{4. 答案生成}
生成采用模板化策略:
\begin{itemize}
\item 若计算器返回 `status=ok` 且通过门控,则输出结构化计算结果与简要解释;
\item 否则回退到 baseline 的占位式生成,并记录 fallback 原因。
\end{itemize}

---

\section{4 实验设置}

\section{数据集与划分}
使用 FinDER 数据集,按官方或既有切分方式划分为 train / dev / test。所有子集与样本格式统一为:

\begin{verbatim}
{
  "qid": "...",
  "query": "...",
  "answer": "...",
  "evidences": [{"text": "...", "doc_id": null, "meta": {}}],
  "meta": {}
}
\end{verbatim}

\section{子集定义}
\begin{itemize}
\item \textbf{complex_dev}:满足任一条件即进入子集:
  - 多证据(evidence ≥ 2)
  - 查询包含 ≥2 年份
  - 查询含比较/变化关键词(vs/compare/yoy/增长率 等)
  - 查询含数值与年份组合
\item \textbf{numeric_dev}:查询或答案含数值/百分号/同比/差值/倍数关键词。
\end{itemize}

\section{评价指标}
\begin{itemize}
\item \textbf{检索指标}:Recall@k、MRR@k、evidence_hit@k
\item \textbf{数值指标}:Numeric-EM、相对误差(RelErr)、覆盖率(Coverage)
\item \textbf{不确定匹配比例}:当 doc_id/evidence_id 缺失时,回退到文本匹配并记录比例。
\end{itemize}

\section{关键参数}
\begin{itemize}
\item 检索器:稀疏(BM25)+ 稠密(sentence-transformers)+ 混合(alpha=0.5)
\item 多步检索(best):max_steps=2, top_k_each_step=10, merge_strategy=maxscore
\item 计算器门控(best):min_conf=0.2, allow_task_types=[]
\end{itemize}

所有实验参数与最终配置均在 `outputs/<run_id>/config.resolved.yaml` 中可复现追溯。

---

\section{5 实验结果}

本节直接引用 Step6 自动生成的结果表与指标文件(见 `docs/TABLE_MAIN.md`、`docs/TABLE_NUMERIC.md`),并给出关键对照结论。对应 run_id 见 `configs/step6_experiments.yaml`。

\section{1) 检索效果(full dev / complex dev)}

主表见:`docs/TABLE_MAIN.md`

关键结论(complex dev):
\begin{itemize}
\item \textbf{baseline(post-ft) vs best multistep}
  - Recall@10:0.3909465 → 0.3909465(持平)
  - MRR@10:0.2960138 → 0.2960873(+0.00007)
\end{itemize}

retriever 微调带来的整体提升(full dev):
\begin{itemize}
\item pre-ft baseline → post-ft baseline:Recall@10 从 0.3246 提升到 0.3772(+0.0526)
\end{itemize}

\section{2) 数值题表现(numeric dev)}

数值表见:`docs/TABLE_NUMERIC.md`

关键对照(numeric dev):
\begin{itemize}
\item \textbf{baseline(post-ft) vs best calc gate}
  - Numeric-EM:0.3838 → 0.3838(持平)
  - RelErr(mean):683.3536 → 683.3536(持平)
  - Coverage:0.6266 → 0.6266(持平)
\end{itemize}

说明:当前版本计算器门控在 dev 上选择 `allow_task_types=[]`,以避免数值误差回退。因此 numeric 指标未出现回退,但也尚未体现提升。该结果为“安全启用”基线,可在后续提升抽取/计算置信度后再重新开启任务类型。

\section{3) 六组矩阵实验(Step6)}
run_id 对照:
\begin{itemize}
\item pre_ft_baseline: `20260130_234540_ae7cdf_m01`
\item post_ft_baseline: `20260130_234540_ae7cdf_m02`
\item post_ft_multistep_best: `20260130_234540_ae7cdf_m03`
\item post_ft_baseline_calc_best: `20260130_234540_ae7cdf_m04`
\item post_ft_multistep_calc_best: `20260130_234540_ae7cdf_m05`
\item post_ft_multistep_T1_calc_best: `20260130_234540_ae7cdf_m06`
\end{itemize}

详细指标已自动写入对应的 `outputs/<run_id>/summary.json` 与 `docs/TABLE_*.md`。

---

\section{6 错误分析与案例}

本节基于 Step6 的 `error_buckets.py` 统计结果与 multistep traces,给出主要失败类型与典型案例。所有数值均可在 `outputs/<run_id>/error_bucket_stats.json`、`outputs/<run_id>/multistep_traces.jsonl` 中复现。

\section{1) 失败类型概览(自动统计)}

以下为 Step6 六组矩阵实验的自动统计摘要:

\begin{itemize}
\item Run 20260130_234540_ae7cdf_m01: numeric_buckets={'fallback': 570}
\item Run 20260130_234540_ae7cdf_m02: numeric_buckets={'fallback': 570}
\item Run 20260130_234540_ae7cdf_m03: complex_buckets={'max_steps': 45, 'no_gap': 525}
\item Run 20260130_234540_ae7cdf_m04: numeric_buckets={'fallback': 570}
\item Run 20260130_234540_ae7cdf_m05: numeric_buckets={'fallback': 570}; complex_buckets={'max_steps': 45, 'no_gap': 525}
\item Run 20260130_234540_ae7cdf_m06: numeric_buckets={'fallback': 570}; complex_buckets={'max_steps': 46, 'no_gap': 524}
\end{itemize}

解释:
\begin{itemize}
\item \textbf{numeric_buckets=fallback}:由于 Step6 最优门控将 `allow_task_types=[]`,计算器任务被完全关闭,所有样本都回退到 baseline;因此 numeric 失败类型呈现为 fallback。
\item \textbf{complex_buckets=no_gap / max_steps}:多步检索在大部分样本中检测到 gap 并运行至 max_steps;在未发现 gap 的样本中直接停止(no_gap)。
\end{itemize}

\section{2) 典型复杂查询案例(complex_dev)}

\textbf{qid}: `8c8c8c34`

\textbf{query}: 
> Hasbro (HAS) 2023 one-time charges impact on operating profitability vs historical trends and cap allocation implications.

\textbf{多步检索轨迹摘要}(来自 `outputs/20260130_234540_ae7cdf_m03_ms/multistep_traces.jsonl`):
\begin{itemize}
\item step0: gap=MISSING_ENTITY, gap_conf=1.0, gate_decision=true, stop_reason=CONTINUE
\item step1: gap=MISSING_ENTITY, gap_conf=1.0, stop_reason=MAX_STEPS
\item final_topk_size=10
\end{itemize}

\textbf{候选证据(部分 chunk_id)}:
\begin{itemize}
\item 008beea7_e0_c0
\item 8c8c8c34_e0_c2
\item f8aec91a_e0_c1
\item 8caea930_e0_c2
\item caa865da_e0_c3
\end{itemize}

\textbf{分析}:
该问题涉及“对比历史趋势 + 一次性费用影响”,属于复杂查询。多步检索识别到 entity/compare 型 gap,但 refined query 与原查询高度相似,导致第二步检索未引入新证据(newly_added_chunk_ids 为空),最终以 MAX_STEPS 停止。该案例反映出 \textbf{refiner 仍偏保守},需要进一步提升 entity 拆分与精细化 query 改写能力。

\section{3) 数值题失败模式(numeric_dev)}

当前版本中计算器通过门控被关闭(allow_task_types=[]),所有数值题回退到 baseline,从而避免 Numeric-EM 下降,但也导致 \textbf{计算器未体现增益}。后续工作需结合更可靠的单位/年份对齐与置信度校准,逐步解除 gate 并验证 Numeric-EM/RelErr 的提升。

---

\section{7 讨论}

\section{1) 多步检索的收益与限制}
多步检索在 complex dev 上未显著提升 Recall@10,但通过门控与合并策略避免了退化。当前的 gap 检测与 query refiner 偏规则化,\textbf{对实体拆分与对齐}仍不够稳定,导致多步检索在部分复杂查询中仅重复或轻度改写查询。

\section{2) 计算器的可控性与覆盖率}
数值抽取与计算模块在未充分校准前,容易出现“算得更多但算错更多”。因此 Step6 中使用 gate 将计算任务类型暂时关闭,保证 numeric 指标不回退。后续工作需要:
\begin{itemize}
\item 引入更鲁棒的单位与实体对齐
\item 更细粒度的置信度打分
\item 限定任务类型并改进 query-based 计算任务识别
\end{itemize}

\section{3) 误差来源}
错误分析显示,复杂查询中主要问题集中在:
\begin{itemize}
\item gap 识别不足(无法稳定抽取比较对象)
\item 合并策略未显著改善证据排序
数值问题中主要问题集中在:
\item 结构化事实不足(缺年份或单位)
\item 计算门控触发率过低
\end{itemize}

\section{4) 可扩展方向}
\begin{itemize}
\item 引入更强的 dense retriever 与领域适配(如金融领域预训练)
\item 使用轻量化的 query rewriting 或规则图谱进行更精准的 gap 解析
\item 扩展计算器支持更多指标、单位与财务结构
\end{itemize}

---

\section{8 结论}

本文构建了一个可复现的金融 RAG 工程体系,覆盖数据规范化、检索、评测与实验编排,并在此基础上实现了多步检索与显式计算模块。实验表明:

\begin{itemize}
\item 检索器微调显著提升了整体检索表现(full dev Recall@10: 0.3246 → 0.3772)。
\item 多步检索在复杂子集上不再造成性能回退,并在 MRR 上呈现轻微提升。
\item 计算器通过门控避免了数值指标退化,为后续提升奠定了稳定基线。
\end{itemize}

未来工作将聚焦于:提升 gap 识别与多步合并策略的有效性、增强数值抽取与单位对齐的鲁棒性,并探索更强的检索与推理模块。

---

\section{附录A 结果表格}

\textbf{表A1 主结果(full dev 与 complex dev)}

\begin{table}[h]
\centering
\begin{tabular}{llllll}
\toprule
label & run_id & full_r10 & full_mrr10 & complex_r10 & complex_mrr10 \\
\midrule
pre_ft_baseline & 20260130_234540_ae7cdf_m01 & 0.3246 & 0.2030 & 0.3457 & 0.2330 \\
post_ft_baseline & 20260130_234540_ae7cdf_m02 & 0.3772 & 0.2601 & 0.3909 & 0.2960 \\
post_ft_multistep_best & 20260130_234540_ae7cdf_m03 & 0.3772 & 0.2601 & 0.3909 & 0.2961 \\
post_ft_baseline_calc_best & 20260130_234540_ae7cdf_m04 & 0.3772 & 0.2601 & 0.3909 & 0.2960 \\
post_ft_multistep_calc_best & 20260130_234540_ae7cdf_m05 & 0.3772 & 0.2601 & 0.3909 & 0.2961 \\
post_ft_multistep_T1_calc_best & 20260130_234540_ae7cdf_m06 & 0.3772 & 0.2601 & 0.3909 & 0.2960 \\
\bottomrule
\end{tabular}
\end{table}

\textbf{表A2 数值题结果(numeric dev)}

\begin{table}[h]
\centering
\begin{tabular}{lllll}
\toprule
label & run_id & num_em & num_rel & num_cov \\
\midrule
pre_ft_baseline & 20260130_234540_ae7cdf_m01 & 0.3791 & 2874.5248 & 0.6202 \\
post_ft_baseline & 20260130_234540_ae7cdf_m02 & 0.3838 & 683.3536 & 0.6266 \\
post_ft_multistep_best & 20260130_234540_ae7cdf_m03 & - & - & - \\
post_ft_baseline_calc_best & 20260130_234540_ae7cdf_m04 & 0.3838 & 683.3536 & 0.6266 \\
post_ft_multistep_calc_best & 20260130_234540_ae7cdf_m05 & 0.3838 & 683.3536 & 0.6266 \\
post_ft_multistep_T1_calc_best & 20260130_234540_ae7cdf_m06 & 0.3838 & 683.3536 & 0.6266 \\
\bottomrule
\end{tabular}
\end{table}

\textbf{表A3 消融结果}

\begin{table}[h]
\centering
\begin{tabular}{llllll}
\toprule
label & run_id & full_r10 & full_mrr10 & complex_r10 & complex_mrr10 \\
\midrule
post_ft_multistep_T1_calc_best & 20260130_234540_ae7cdf_m06 & 0.3772 & 0.2601 & 0.3909 & 0.2960 \\
\bottomrule
\end{tabular}
\end{table}

\section{附录B 典型复杂查询案例}

\subsection{5.4 典型复杂查询案例(3个)}

\textbf{案例1(qid=8c8c8c34)}
\begin{itemize}
\item Query:Hasbro (HAS) 2023 one-time charges impact on operating profitability vs historical trends and cap allocation implications.
\item Gold Answer(摘要):In 2023, Hasbro’s operating result turned from a profit in prior years (407.7 million in 2022 and 763.3 million in 2021) to an operating loss of 1,538.8 million…
\item Step0 Top3:008beea7_e0_c0,8c8c8c34_e0_c2,f8aec91a_e0_c1
\item Step1 Top3:008beea7_e0_c0,f8aec91a_e0_c1,8c8c8c34_e0_c2
\item gap/stop:MISSING_ENTITY / MAX_STEPS,final_topk_size=10
\item 分析:该问题包含对比关系与年份信息,多步检索识别到 gap,但 refined query 与原查询高度相似,导致后续步骤新增证据较少,表明实体拆分与重写仍需加强。
\end{itemize}

\textbf{案例2(qid=52e25ec7)}
\begin{itemize}
\item Query:Impact on net investing cash flows from EUC sale cash inflow offsets vs acquisition outflows, AVGO.
\item Gold Answer(摘要):The $3,485 million inflow from the sale of the EUC business helped to partially offset the significantly higher cash expenditures related to acquisitions…
\item Step0 Top3:506e7d1e_e0_c0,52e25ec7_e0_c0,e4661352_e0_c3
\item Step1 Top3:506e7d1e_e0_c0,52e25ec7_e0_c0,1c47856d_e0_c1
\item gap/stop:MISSING_ENTITY / MAX_STEPS,final_topk_size=10
\item 分析:问题涉及“出售现金流入 vs 并购现金流出”的对比。多步检索能够维持对核心证据的覆盖,但仍未显著扩展证据范围。
\end{itemize}

\textbf{案例3(qid=ed746c33)}
\begin{itemize}
\item Query:Cash flow & cap alloc implications of IRM's ASC 842 storage rev rec vs other lines.
\item Gold Answer(摘要):For its Global Data Center Business, Iron Mountain recognizes storage revenues under ASC 842…
\item Step0 Top3:ed746c33_e0_c0,2a8785e8_e0_c15,a68b8600_e0_c5
\item gap/stop:NO_GAP / NO_GAP,final_topk_size=10
\item 分析:该类问题实体明确、语义集中,单步即可覆盖核心证据,多步检索不会引入额外噪声。
\end{itemize}

\section{参考文献}
[1] 参考文献占位。
\end{document}