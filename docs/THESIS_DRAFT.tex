\documentclass[UTF8]{ctexart}
\usepackage{geometry}
\usepackage{booktabs}
\usepackage{longtable}
\usepackage{hyperref}
\geometry{a4paper, left=2.5cm, right=2.5cm, top=2.5cm, bottom=2.5cm}

\title{基于多步检索与显式计算的金融RAG系统研究}
\author{(待填)}
\date{\today}

\begin{document}
\maketitle

\section*{摘要}
金融问答(FinDER)任务中的复杂查询常涉及多年份、多实体与显式数值计算。传统单步检索 + 占位式生成在复杂问题上易出现证据覆盖不足与算术错误。本文构建了一个可复现的金融 RAG 系统,在无外部 LLM API 的约束下,引入规则驱动的多步检索与显式计算器,并通过门控策略与系统化调参避免性能回退。实验结果表明:检索器微调显著提升整体检索表现(full dev Recall@10: 0.3246 → 0.3772);多步检索在复杂子集上维持不退化且 MRR 略有提升;计算器通过门控确保数值指标不下降,为后续提升奠定稳定基线。

\section{引言}
金融问答(FinDER)场景中的查询往往具有高信息密度与强对比/计算需求:同一问题可能同时涉及多个年份、实体、指标,并要求对证据进行对齐与推理。传统单步检索 + 占位式生成容易在复杂问题上出现证据覆盖不足与算术错误。

为此,本文围绕可复现的金融 RAG(Retrieval-Augmented Generation)系统,构建并验证了一个分层可控的工程方案:在强约束(无外部 LLM API)的条件下,引入规则驱动的多步检索与显式计算器,并通过系统化调参与门控机制避免性能回退。

本文贡献如下:
\begin{itemize}
  \item 可复现工程框架:建立从数据规范化、检索、评测到实验编排的一体化体系,所有实验产出统一落盘,可审计、可回滚。
  \item 多步检索与门控策略:实现 gap 检测、合并策略与停止规则,构建可控的多步检索循环,保障复杂查询的证据覆盖。
  \item 显式计算器:在证据抽取与单位/年份校验基础上进行程序化计算,并通过门控阈值降低算错风险。
  \item 系统化调参与对照分析:输出 full dev / complex dev / numeric dev 的对照与消融结果,支持论文级结果表格与错误分析。
\end{itemize}

\section{相关工作}
\subsection{金融问答与金融文本理解}
金融 QA 数据集与任务通常聚焦于财报、公告、研报等长文档环境中的事实与数值问答。FinDER 等数据集强调对证据的精确定位与多字段对齐,对检索与推理能力要求较高。

\subsection{检索增强生成(RAG)}
RAG 框架通过检索模块提供高相关证据,再由生成模块构造答案。近期研究关注密集检索、稀疏检索与混合检索的结合,以及检索器微调对下游 QA 的传导效果。

\subsection{多跳/多步检索与推理}
多跳检索强调通过多轮检索逐步完善证据覆盖,常见于复杂比较问题与跨文档推理问题。本文的多步检索采用规则驱动的 gap 检测与可控门控策略,以保证解释性和稳定性。

\subsection{显式数值推理与计算}
数值类问题的核心挑战是单位、年份与数值的对齐。显式计算器通过结构化抽取与程序化计算降低算术错误,并配合门控策略避免错误传播。

\section{方法}
本文系统由四个核心模块组成:查询理解、多步检索推理、证据整合与计算、答案生成。

\subsection{查询理解}
通过规则与词典对查询进行初步解析,包括年份识别、比较关系识别、数值提示词识别等。该模块用于驱动多步检索的 gap 类型判断与后续计算任务类型预测。

\subsection{多步检索推理}
多步检索在每一步使用当前查询进行检索,随后基于 gap 检测决定是否继续检索。核心机制包括:Gap Detector、Gate(门控)、Merge Strategy、Stop Criteria。\newline
Step6 最优多步配置(dev):max\_steps=2, top\_k\_each\_step=10, novelty\_threshold=0.0, stop\_no\_new\_steps=1, merge\_strategy=maxscore, gate.min\_gap\_conf=0.3。

\subsection{证据整合与计算}
从证据文本中提取数值、单位、年份,并标注 inferred\_year 与 confidence;计算器支持 YoY/差值/占比/倍数等任务,对单位不一致、年份缺失、候选冲突等情况进行拒算并记录原因。Step6 最优门控(numeric dev):min\_conf=0.2,allow\_task\_types=[]。

\subsection{答案生成}
若计算器返回 status=ok 且通过门控,则输出结构化计算结果与简要解释;否则回退到 baseline 的占位式生成,并记录 fallback 原因。

\section{实验设置}
使用 FinDER 数据集,划分 train/dev/test。子集 complex\_dev 与 numeric\_dev 按规则构建。评价指标包括 Recall@k、MRR@k、Numeric-EM、RelErr 与 Coverage。所有参数与配置写入 outputs/<run\_id>/config.resolved.yaml。

\section{实验结果}
\subsection{检索效果}
complex dev 关键对照(baseline vs best multistep):Recall@10 0.3909465 → 0.3909465;MRR@10 0.2960138 → 0.2960873。
\subsection{数值题效果}
numeric dev 关键对照(baseline vs best calc gate):Numeric-EM 0.3838 → 0.3838;RelErr(mean) 683.3536 → 683.3536;Coverage 0.6266 → 0.6266。

\section{误差分析与讨论}
复杂查询中主要问题为 gap 识别不足与合并策略收益有限;数值题中主要问题为结构化事实不足与计算门控触发率偏低。典型复杂查询 qid=8c8c8c34 在多步检索中出现 gap=MISSING\_ENTITY,步骤间证据重合较多,最终 MAX\_STEPS 停止。

\section{结论}
本文构建了可复现的金融 RAG 工程体系,并验证多步检索与显式计算的可控性。检索器微调显著提升检索效果,多步检索避免复杂子集退化,计算器通过门控避免数值指标回退。未来将提升 gap 识别能力与计算器鲁棒性。

\end{document}
